%%%%%%%%%%%%%%%%%%%%%%%%%%%%%%%%%%%%%%%%%
% Beamer Presentation
% LaTeX Template
% Version 1.0 (10/11/12)
%
% This template has been downloaded from:
% http://www.LaTeXTemplates.com
%
% License:
% CC BY-NC-SA 3.0 (http://creativecommons.org/licenses/by-nc-sa/3.0/)
%
%%%%%%%%%%%%%%%%%%%%%%%%%%%%%%%%%%%%%%%%%

%----------------------------------------------------------------------------------------
%	PACKAGES AND THEMES
%----------------------------------------------------------------------------------------

\documentclass{beamer}
\usepackage[utf8]{inputenc}

\mode<presentation> {

% The Beamer class comes with a number of default slide themes
% which change the colors and layouts of slides. Below this is a list
% of all the themes, uncomment each in turn to see what they look like.

%\usetheme{default}
%\usetheme{AnnArbor}
%\usetheme{Antibes}
%\usetheme{Bergen}
%\usetheme{Berkeley}
%\usetheme{Berlin}
%\usetheme{Boadilla}
%\usetheme{CambridgeUS}
\usetheme{Copenhagen}
%\usetheme{Darmstadt}
%\usetheme{Dresden}
%\usetheme{Frankfurt}
%\usetheme{Goettingen}
%\usetheme{Hannover}
%\usetheme{Ilmenau}
%\usetheme{JuanLesPins}
%\usetheme{Luebeck}
%\usetheme{Madrid}
%\usetheme{Malmoe}
%\usetheme{Marburg}
%\usetheme{Montpellier}
%\usetheme{PaloAlto}
%\usetheme{Pittsburgh}
%\usetheme{Rochester}
%\usetheme{Singapore}
%\usetheme{Szeged}
%\usetheme{Warsaw}

% As well as themes, the Beamer class has a number of color themes
% for any slide theme. Uncomment each of these in turn to see how it
% changes the colors of your current slide theme.

%\usecolortheme{albatross}
%\usecolortheme{beaver}
%\usecolortheme{beetle}
%\usecolortheme{crane}
%\usecolortheme{dolphin}
%\usecolortheme{dove}
%\usecolortheme{fly}
%\usecolortheme{lily}
\usecolortheme{orchid}
%\usecolortheme{rose}
%\usecolortheme{seagull}
%\usecolortheme{seahorse}
%\usecolortheme{whale}
%\usecolortheme{wolverine}

%\setbeamertemplate{footline} % To remove the footer line in all slides uncomment this line
%\setbeamertemplate{footline}[page number] % To replace the footer line in all slides with a simple slide count uncomment this line

\setbeamertemplate{navigation symbols}{} % To remove the navigation symbols from the bottom of all slides uncomment this line
}

\usefonttheme{structurebold}

\usepackage{graphicx} % Allows including images
\usepackage{booktabs} % Allows the use of \toprule, \midrule and \bottomrule in tables
\usepackage{alltt}
\usepackage{sty/pythonhighlight}
\usepackage{listings}
\usepackage{dirtree}

\newcommand{\bashcmd}[1]{\vspace{1mm}\hspace{5mm}\texttt{#1\\}\vspace{1mm}}

%----------------------------------------------------------------------------------------
%	TITLE PAGE
%----------------------------------------------------------------------------------------

\title[Flask]{Flask} % The short title appears at the bottom of every slide, the full title is only on the title page

\author{Marcin Jenczmyk} % Your name
\institute[Clearcode] % Your institution as it will appear on the bottom of every slide, may be shorthand to save space
{
Clearcode \\ % Your institution for the title page
\medskip
\href{mailto:m.jenczmyk@clearcode.cc}{\nolinkurl{m.jenczmyk@clearcode.cc}} % Your email address
}
\date{27/05/2017} % Date, can be changed to a custom date

\begin{document}

\begin{frame}
\titlepage % Print the title page as the first slide
\end{frame}

\begin{frame}
\frametitle{Overview} % Table of contents slide, comment this block out to remove it
\tableofcontents % Throughout your presentation, if you choose to use \section{} and \subsection{} commands, these will automatically be printed on this slide as an overview of your presentation
\end{frame}

%----------------------------------------------------------------------------------------
%	PRESENTATION SLIDES
%----------------------------------------------------------------------------------------

\section{requirements.txt}

\begin{frame}
  \frametitle{requirements.txt}

  Requirements file is a plaintext file listing Python \texttt{pip} dependencies for a project.

  \vspace{3mm}

  \pause To install dependencies from \texttt{requirements.txt} into current Python envrinoment run\\
  \bashcmd{doctor@TARDIS:$\sim$\$ pip install -r requirements.txt}

  \vspace{3mm}

  \pause To save list of Python packages installled in current Python envrinoment into \texttt{requirements.txt} file run\\
  \bashcmd{doctor@TARDIS:$\sim$\$ pip freeze > requirements.txt}
\end{frame}

\begin{frame}
  \frametitle{requirements.txt}

  \begin{figure}
    \fbox{\begin{minipage}{\textwidth}
      \begin{alltt}
        appdirs==1.4.3\\
        click==6.7\\
        Flask==0.12.2\\
        itsdangerous==0.24\\
        Jinja2==2.9.6\\
        MarkupSafe==1.0\\
        packaging==16.8\\
        pyparsing==2.2.0\\
        six==1.10.0\\
        Werkzeug==0.12.2
      \end{alltt}
    \end{minipage}}
    \caption{A sample requirements file.}
  \end{figure}
\end{frame}

\section{Hello there!}

\begin{frame}
  \frametitle{Hello there!}

  Flask is a microframework for Python for a web development. \pause Some useful resources:
  \begin{itemize}
    \item \url{http://flask.pocoo.org/}
    \item \url{http://flask.pocoo.org/docs/latest/quickstart/}
    \item \url{http://flask.pocoo.org/extensions/}
    \item \url{https://blog.miguelgrinberg.com/post/the-flask-mega-tutorial-part-i-hello-world}
  \end{itemize}

  \vspace{3mm}

  \pause \bashcmd{doctor@TARDIS:$\sim$\$ pip install Flask}

\end{frame}

\begin{frame}
  \frametitle{Hello there!}

  \begin{figure}
    \inputpython{examples/hello_v1/hello.py}{1}{10}
    \caption{A simple Flask app (see \texttt{hello\_v1/hello.py}).}
  \end{figure}
\end{frame}

\begin{frame}
  \frametitle{Hello there!}

  To run a Flask application run \\
  \bashcmd{doctor@TARDIS:$\sim$\$ python hello.py}

  \pause

  \begin{block}{Debug mode}
    It's easier to debug application behaviour in debug mode - to do this add \pyth{app.debug \= True}
    in your Python code or export \texttt{FLASK\_DEBUG} envrinoment variable \\
    \bashcmd{doctor@TARDIS:$\sim$\$ export FLASK\_DEBUG=1}
  \end{block}

  \pause

  \begin{block}{Warning}
    Debug mode should be never used on a production!
  \end{block}
\end{frame}

\section{HTTP methods}

\begin{frame}
  \frametitle{HTTP methods}

  To use a GET HTTP method put a \pyth{<type:arg>} in a view URL, where \pyth{type} can be either \texttt{string},
  \texttt{int}, \texttt{float}, \texttt{path}, \texttt{any} (any of listed before) or \texttt{uuid}.

  \vspace{3mm}

  \begin{figure}
    \inputpython{examples/hello_v2/hello.py}{7}{15}
    \caption{\centering{Sample view with GET parameter (see \texttt{hello\_v2/hello.py}).}}
  \end{figure}
\end{frame}

\begin{frame}
  \frametitle{HTTP methods}

  To use POST HTTP method one has to enable it in a view decorator.

  \vspace{3mm}

  \begin{figure}
    \inputpython{examples/hello_v2/hello.py}{18}{26}
    \caption{\centering{Sample view handling POST parameter (see \texttt{hello\_v2/hello.py}).}}
  \end{figure}

\end{frame}

\begin{frame}
  \frametitle{HTTP methods}

  \begin{block}{Remark}
    One can get URL to view by its name using \pyth{url_for} function, ex. one can get login view URL by calling
    \pyth{url_for('login')}.
  \end{block}
\end{frame}

\section{Templates}

\begin{frame}
  \frametitle{Jinja2}

  Rendering entire HTML markup for a webpage by writing strings to be returned by views would be tiresome - there
  is a \href{http://jinja.pocoo.org/}{Jinja2} engine built in Flask to enable rendering HTML templates.

  \vspace{3mm} \pause

  One can render template using \pyth{render_template} function, \textbf{Flask will be looking for templates
  in the \texttt{templates} directory, located at the same path as Python application file!}

  \dirtree{%
    .1 /.
    .2 hello.py.
    .2 templates.
  }
\end{frame}

\begin{frame}
  \frametitle{Jinja2}
  \begin{figure}
    \inputpython{examples/hello_v3/hello.py}{6}{18}
    \caption{\centering{Sample view rendering Jinja template (see \texttt{hello\_v3/hello.py}).}}
  \end{figure}

\end{frame}

\begin{frame}
  \frametitle{Jinja2}

  \begin{figure}
    \lstinputlisting[language=HTML, frame=single]{examples/hello_v3/templates/index.html}
    \caption{\centering{Jinja template for hello\_3 example (see \texttt{hello\_v3/templates/index.html}).}}
  \end{figure}
\end{frame}

\begin{frame}
  \frametitle{Jinja2}
  \begin{figure}
    \lstinputlisting[language=HTML, frame=single]{examples/hello_v4/templates/base.html}
    \caption{\centering{
      Jinja templates can be extended (see \texttt{index.html} and \texttt{base.html}
      - on next slide - in \texttt{hello\_v4/templates/}).
    }}

  \end{figure}

\end{frame}

\begin{frame}
  \frametitle{Jinja2}

  \lstinputlisting[language=HTML, frame=single]{examples/hello_v4/templates/index.html}
\end{frame}

\begin{frame}
  \frametitle{Jinja2}
  \begin{figure}
    \lstinputlisting[language=HTML, frame=single, firstline=1, lastline=7]{examples/loops.html}
    \lstinputlisting[language=HTML, frame=single, firstline=9, lastline=11]{examples/loops.html}
    \caption{\centering{Jinja templates allow for using \texttt{for} loops and \texttt{if} commands.}}
  \end{figure}


\end{frame}

\section{Static files}

\begin{frame}
  \frametitle{Static files}

  Dynamic web applications also need static files. They are going to be searched for in \texttt{static} directory
  (but on production envrinoment server should handle them); to generate URLs for static files, use the special
  \texttt{static} endpoint name.

  \vspace{3mm} \pause

  To generate URL for a static file use \pyth{url_for} function, ex. \pyth{url_for('static', filename='style.css')}.

  \dirtree{%
    .1 /.
    .2 hello.py.
    .2 templates.
    .2 static.
  }
\end{frame}

\begin{frame}
  \frametitle{Static files}

  \begin{figure}
    \inputpython{examples/hello_v5/hello.py}{1}{9}
    \caption{\centering{Python code for \texttt{hello\_v5} example app.}}
  \end{figure}
\end{frame}

\begin{frame}
  \frametitle{Static files}

  \begin{figure}
    \lstinputlisting[language=HTML, frame=single, firstline=1, lastline=8]{examples/hello_v5/templates/index.html}
    \caption{\centering{Jinja template rendering static content (see \texttt{hello\_v5/templates/index.html}, continuation on next slide).}}
  \end{figure}

\end{frame}

\begin{frame}
  \frametitle{Static files}

  \begin{figure}
    \lstinputlisting[language=HTML, frame=single, firstline=10, lastline=14]{examples/hello_v5/templates/index.html}
    \caption{\centering{Jinja template rendering static content, continuation (see \texttt{hello\_v5/templates/index.html}).}}
  \end{figure}

\end{frame}

\begin{frame}
\frametitle{References}
\footnotesize{
\begin{thebibliography}{99} % Beamer does not support BibTeX so references must be inserted manually as below
  \bibitem[Armin, 2017]{p1} Armin Ronacher (2017)
  \newblock \url{http://flask.pocoo.org/}
  \bibitem[Armin, 2008]{p2} Armin Ronacher (2008)
  \newblock \url{http://jinja.pocoo.org/docs/2.9/}

\end{thebibliography}
}
\end{frame}

\end{document}
